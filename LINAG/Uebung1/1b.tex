\documentclass{article}

\usepackage[utf8]{inputenc}
\usepackage[T1]{fontenc}
\usepackage[ngerman]{babel}
\usepackage{amsmath, amssymb}
\usepackage{xcolor}
\usepackage{booktabs}
\usepackage{fancyhdr}
\usepackage{enumitem}

\pagestyle{fancy}
\fancyhead{}
\fancyfoot{}
\fancyhead[R]{\thepage}
\renewcommand{\headrulewidth}{0.4pt}

\begin{document}

\section*{Aufgabe 1:b}
Entscheiden Sie, ob folgende Aussagen \textbf{wahr} oder \textbf{falsch} sind, wenn $A, B, \dots$ Aussagen bezeichnen, die wahr oder falsch sein können:

\subsection*{(i) $A \lor (\neg A)$}

\textbf{Wahrheitstafel:}

\[
    \begin{array}{c|c|c}
        A & \neg A & A \lor (\neg A) \\
        \midrule
        W & F      & W               \\
        F & W      & W               \\
    \end{array}
\]

\textbf{Ergebnis:} Die Aussage ist in allen Fällen \textbf{wahr}.
$\Rightarrow$ \textbf{Tautologie.}

\bigskip
\subsection*{(ii) \(\neg B \land (A \Rightarrow B)\)}

Dies ist die reine Konjunktion (ohne äußere Implikation). Erinnerung: \(A\Rightarrow B \equiv \neg A \lor B\).

\textbf{Wahrheitstafel:}

\[
    \begin{array}{c|c|c|c}
        A & B & A \Rightarrow B & \neg B \land (A \Rightarrow B) \\
        \midrule
        W & W & W               & F                              \\
        W & F & F               & F                              \\
        F & W & W               & F                              \\
        F & F & W               & W                              \\
    \end{array}
\]

\textbf{Ergebnis:} Die Formel ist in manchen Belegungen \textbf{wahr} (z.\,B. \(A=F,B=F\)), in anderen \textbf{falsch}.
$\Rightarrow$ \textbf{keine Tautologie}, aber \textbf{erfüllbar} (kontingent).

\bigskip
\subsection*{(iii) Assoziativität: $A \land (B \land C) \Leftrightarrow (A \land B) \land C$}

\textbf{Wahrheitstafel:}

\[
    \begin{array}{c|c|c|c|c|c}
        A & B & C & A \land (B \land C) & (A \land B) \land C & \text{Äquivalent?} \\
        \midrule
        W & W & W & W                   & W                   & W                  \\
        W & W & F & F                   & F                   & W                  \\
        W & F & W & F                   & F                   & W                  \\
        F & W & W & F                   & F                   & W                  \\
        F & F & F & F                   & F                   & W                  \\
    \end{array}
\]

\textbf{Ergebnis:} Beide Seiten sind stets gleich.
$\Rightarrow$ \textbf{wahr (Tautologie)} — Assoziativgesetz gilt.

\bigskip
\subsection*{(iv) Korrigierte Angabe:}
\[
    C \lor (\neg\neg C \land A \land (B \lor C)) \;\Leftrightarrow\; \neg\neg C \land (C \lor (A \land B))
\]

\textbf{Wahrheitstafel:}

\[
    \begin{array}{c|c|c|c|c|c|c|c}
        A                                                                                     & B & C & \neg\neg C & B\lor C & \neg\neg C\land A\land(B\lor C) & \text{links} & \text{rechts} \\
        \midrule
        W                                                                                     & W & W & W          & W       & W                               & W            & W             \\
        W                                                                                     & W & F & F          & W       & F                               & F            & F             \\
        W                                                                                     & F & W & W          & W       & W                               & W            & F             \\
        W                                                                                     & F & F & F          & F       & F                               & F            & F             \\
        F                                                                                     & W & W & W          & W       & F                               & W            & F             \\
        F                                                                                     & W & F & F          & W       & F                               & F            & F             \\
        F                                                                                     & F & W & W          & W       & F                               & W            & F             \\
        F                                                                                     & F & F & F          & F       & F                               & F            & F             \\
        \midrule
        \multicolumn{7}{r}{\text{Äquivalenz (links $\Leftrightarrow$ rechts) in jeder Zeile}} & W                                                                                             \\
    \end{array}
\]

(Darstellung: für jede Belegung sind die Werte von \emph{links} und \emph{rechts} gleich; die Äquivalenz ist in allen 8 Fällen wahr.)

\textbf{Ergebnis:} Beide Seiten haben in \emph{jeder} Belegung denselben Wahrheitswert.
$\Rightarrow$ \textbf{Tautologie} (wahr).

\bigskip
\hrule
\bigskip
\textbf{Zusammenfassung:}

\[
    \begin{array}{l|c}
        \text{Aussage} & \text{Wahrheitswert}     \\
        \midrule
        (i)            & \text{wahr (Tautologie)} \\
        (ii)           & \text{falsch}            \\
        (iii)          & \text{wahr (Tautologie)} \\
        (iv)           & \text{falsch}            \\
    \end{array}
\]

\end{document}
